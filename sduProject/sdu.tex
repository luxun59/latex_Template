\documentclass[12pt, a4paper, oneside]{ctexart}
\usepackage{amsmath, amsthm, amssymb, appendix, bm, graphicx, hyperref, mathrsfs}
\usepackage{fancyhdr}%导入fancyhdr包
\usepackage{geometry} 
\usepackage{booktabs} %三行线表格用
\pagenumbering{Alph}%设置页码格式,大写字母标页
\pagestyle{fancy}
%%%%% 正文页眉设置
%清除页眉页脚
\fancyhf{} 
\fancyhead[C]{\songti\zihao{-5} 山东大学本科毕业论文(设计)}
\fancyfoot[C]{\thepage}
%显示总页数
% \usepackage{lastpage}
% \cfoot{\thepage / \pageref{LastPage}}
%设置页眉页脚线宽
\renewcommand{\headrulewidth}{0.1mm}
\renewcommand{\footrulewidth}{0.1mm}

% 摘要和目录页的页眉页脚
\fancypagestyle{plain} % 定义一个名为 plain 的页眉页脚样式
{
\fancyhf{} 
\fancyhead{} 
\fancyfoot[C]{\thepage} % 只显示号码,RO LE 根据情况进行修改
\renewcommand{\headrulewidth}{0pt}  % 页眉线粗细
\renewcommand{\footrulewidth}{0pt}  % 页脚线粗细
}

%=================设置章节标题格式==================
\ctexset{
	%一级标题:三号黑体加粗,居中对齐,段前0.8段后0.5行,标题编号和标题名之间空1格。
	section={
		name = {},		
		number = {\arabic{section}},
		format = {\heiti \centering \bfseries \zihao{3}},
		aftername = \hspace{9pt},
		beforeskip = 17.7pt,
		afterskip = 11pt,
		fixskip = true,
	},
	subsection={
		%二级标题:左对齐,四号黑体加粗,段前段后间距0.5行,标题编号和标题内容空1格
		number = {\arabic{section}.\hspace{2pt}\arabic{subsection}},
		format = {\heiti \raggedright \bfseries \zihao{4}},
		aftername = \hspace{8pt},
		beforeskip = 9.7pt,
		afterskip = 9.7pt,
		fixskip = true,
	},
	subsubsection={
		%三级标题:小四号黑体加粗,左对齐,段前段后0.5行,标题编号和标题内容空1格。
		number = {\arabic{section}.\hspace{2pt}\arabic{subsection}.\hspace{2pt}\arabic{subsubsection}},
		format = {\heiti  \raggedright \bfseries \zihao{-4}},
		aftername = \hspace{9pt},
		beforeskip = 8.3pt,
		afterskip = 8.3pt,
		fixskip = true,
	}
}

\linespread{1.0}
\geometry{left=3.0cm, right=3.0cm, top=2.5cm, bottom=2.5cm}
\newtheorem{theorem}{定理}[section]
\newtheorem{definition}[theorem]{定义}
\newtheorem{lemma}[theorem]{引理}
\newtheorem{corollary}[theorem]{推论}
\newtheorem{example}[theorem]{例}
\newtheorem{proposition}[theorem]{命题}
\renewcommand{\abstractname}{\Large\textbf{摘要}}

\begin{document}

\thispagestyle{empty}

\begin{figure}[t]
    \centering
    \includegraphics[width=11cm]{logo3.png}
\end{figure}

\vspace*{\fill}
    \begin{center}
        \Huge\textbf{大作业标题}
    \end{center}
\vspace*{\fill}

\begin{table}[b]
    \centering
    \large
    \begin{tabular}{ll}
    \textbf{课程:} & 摸鱼学导论 \\
    \textbf{姓名:} & luxun \\
    \textbf{班级:} & 自动化19.1 \\
    \textbf{时间:} & \today \\
    \end{tabular}
\end{table}


\newpage

\thispagestyle{empty}
\begin{abstract}
    这里是摘要. 
    \par\textbf{关键词:}这里是关键词; 这里是关键词. 
\end{abstract}

\newpage
% \tocloftpagestyle{plain}
\thispagestyle{plain}
\pagenumbering{Roman}
\setcounter{page}{1}
\tableofcontents

\newpage
\setcounter{page}{1}
\pagenumbering{arabic}

\section{一级标题}

    \subsection{二级标题}

        \subsubsection{三级标题}

这里是正文. 


\section{一级标题(有关表格)}

    \subsection{普通表格}

    \begin{table}[h!]
        \begin{center}
            \caption{\songti \zihao{5}\bfseries 示例表格}
            \begin{tabular}{l|c|r} % <-- Alignments: 1st column left, 2nd middle and 3rd right, with vertical lines in between
            \toprule
            \textrm{Value 1} & \textrm{Value 2} & \textrm{Value 3}\\
            $\alpha$ & $\beta$ & $\gamma$ \\
            \hline
            \multicolumn{2}{c|}{12} & a\\ % <-- Combining two cells with alignment c| and content 12.
            \hline
            1 & 1110.1 & b\\
            2 & 10.1 & c\\
            3 & 23.113231 & d\\
            \bottomrule
            \end{tabular}
        \end{center}
    \end{table}

    \subsection{三线表格}

\begin{table}[h!]
    \begin{center}
      \caption{\songti \zihao{5}\bfseries 三线表格}
      \label{tab:table1}
        \begin{tabular}{lcr}
        \toprule
        \textbf{Value 1} & \textbf{Value 2} & \textbf{Value 3}\\
        $\alpha$ & $\beta$ & $\gamma$ \\
        \hline
        \multicolumn{2}{c}{12} & a\\ % <-- Combining two cells with alignment c| and content 12.
        \hline
        2 & 10.1 & b\\
        3 & 23.113231 & c\\
        \checkmark & $\times$ & \\
        \bottomrule
      \end{tabular}
    \end{center}
\end{table}



\newpage

\begin{thebibliography}{99}
    \bibitem{a}作者. \emph{文献}[M]. 地点:出版社,年份.
    \bibitem{b}作者. \emph{文献}[M]. 地点:出版社,年份.
\end{thebibliography}

\newpage

\begin{appendices}
    \renewcommand{\thesection}{\Alph{section}}
    \section*{\heiti \zihao{-2}附\qquad录}
        这里是附录. 
    \addappheadtotoc
\end{appendices}

\end{document}